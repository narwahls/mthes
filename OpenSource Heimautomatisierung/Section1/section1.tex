% !TEX encoding = latin1

\section{Einf�hrung}

\subsection{Einleitung}
Heimautomatisierung beschreibt im Wesentlichen die kontrollierte, vordefinierte Abarbeitung von gestellten Aufgaben im Bezug auf Prozesse im eigenen Haus, in der eigenen Wohnung. Im allgemeinen ist dabei nicht definiert, wie solche Prozesse aussehen oder was ein Prozess enthalten muss. Dies kann von kleinsten Aufgaben ausgehen und bei komplexen Steuermechanismen enden. 

Ein grundlegendes Problem der Heimautomatisierung ist dabei nach wie vor eine hohe Kostenspanne, die nicht unbedingt durch eine gro�e Anzahl von Personen angenommen wird. G�nstige Alternative beschr�nken sich dabei oft auf ein kleines, einfach zu implementierendes System mit beschr�nkter Erweiterbarkeit. So sind hier oft eingeschr�nkte Funktionsumf�nge hinzunehmen.

Sollten kommerzielle Produkte f�r die Heimautomatisierung in Erw�gung gezogen werden, so kann hier bereits auf ein breites Spektrum an Herstellern zur�ckgegriffen werden. 