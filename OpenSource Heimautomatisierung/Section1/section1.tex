% !TEX encoding = latin1

\section{Einf�hrung}

\subsection{Einleitung}
Drahtlose Datennetzwerke erfahren einer immer h�heren Beliebtheit und einer steigenden Akzeptanz. In vielen Bereichen des Lebens treffen wir auf Vertreter diesen Bereiches. Bluetooth und WLAN sind nur 2 Vertreter davon. Doch deren Nutzung bedarf in Verschiedenen Bereichen ein hohes Ma� an Aufmerksamkeit. Wir treffen auf Sie in Caf�s, an Bahnh�fen, auf der Arbeit. Und allen diesen Orten gibt es den Aspekt der Sicherheit auf den geachtet werden muss. Und doch werden solche Angebote immer weiter ausgebaut. Und der Ausbau macht auch nicht vor Bildungseinrichtungen halt.



\subsection{Problemstellung}
Innerhalb von Bildungseinrichtungen gelten besondere Anforderungen. Definiert durch Jugendschutzgesetz und anderen Richtlinien ist bei Aufbauen eines WLAN Netzwerkes innerhalb einer Bildungseinrichtung verschiedenes zu beachten. Durch nicht einhalten solcher Richtlinien ist ein Scheitern des Projektes vordefiniert.


\subsection{Zielsetzung}
In dieser Seminararbeit soll ein m�glicher Aufbau eines WLAN-Netzwerkes innerhalb einer Schule Aufgezeigt werden. Nicht nur wie ein WLAN Netzwerk im allgemeinen 

